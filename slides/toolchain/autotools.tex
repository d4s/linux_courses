\begin{frame}
 \frametitle{Иерархия инструментов поддержки проекта}
 \begin{itemize}
   \item \texttt{gcc -c , gcc -o program file1.o file2.o}
     \pause
   \item \texttt{make programm}
     \begin{enumerate}
       \item[Цель] Автоматизация процесса сборки
      \end{enumerate}
     \pause
   \item \texttt{./configure; make; sudo make install}
     \begin{enumerate}
       \item[Цель] Переносимость между платформами
      \end{enumerate}
     \pause
   \item \texttt{autoreconf, automake, libtool, etc.}
     \begin{enumerate}
       \item[Цель] Автоматизация создания переносимых приложений
      \end{enumerate}
 \end{itemize}
\end{frame}

\begin{frame}
  \frametitle{Типичная схема установки пакета соответсвующего стандартам GNU}
  \begin{itemize}
    \item \texttt{configure}, возможно \texttt{configure <options>}
    \item \texttt{make}
    \item \texttt{sudo make install} 
   \end{itemize}
\end{frame}

\begin{frame}
\frametitle{Упрощенная схема работы GNU configure}

%This slide is copyright Alexandre Duret-Lutz, Laboratoire de Recherche et Développement de l'Epita 
%Distributed under CC BY-SA 2.0 license
%Minor changes Yury Adamov, EPAM 
\begin{center}
\begin{tikzpicture}

\dgfile (Makefilein) at (1,5) {\filenamew{Makefile.in}};
\dgfile (src/Makefilein) at (4,5) {\filenamew{src/Makefile.in}};
\dgfile (confighin) at (7,5) {\filenamew{config.h.in}};
\dgfile (configure) at (4,3) {\filenamew{configure}};
\uncover<2->{
\bfile (Makefile) at (1,1) {\filenamew{Makefile}};
\bfile (src/Makefile) at (4,1) {\filenamew{src/Makefile}};
\bfile (configh) at (7,1) {\filenamew{config.h}};

\arr (Makefilein) .. controls +(-90:1cm) and +(180:3cm) .. (configure);
\arr (src/Makefilein) .. controls +(-90:1cm) and +(175:3cm) .. (configure);
\arr (confighin) .. controls +(-90:1cm) and +(170:3cm) .. (configure);

\arr (configure) .. controls +(-10:2.5cm) and +(80:1cm) .. (Makefile);
\arr (configure) .. controls +(-5:2.2cm) and +(85:1cm) .. (src/Makefile);
\arr (configure) .. controls +(0:2cm) and +(90:1cm) .. (configh);
}
\end{tikzpicture}
\end{center}
\footnotesize{ Взято из http://www.lrde.epita.fr/~adl/autotools.html}
\end{frame}

\begin{frame}
\frametitle{Реальная схема работы configure}


\begin{center}
\begin{tikzpicture}

%This slide is copyright Alexandre Duret-Lutz, Laboratoire de Recherche et Développement de l'Epita 
%Distributed under CC BY-SA 2.0 license
%Minor changes Yury Adamov, EPAM 
\dgfile (Makefilein) at (1,5) {\filenamew{Makefile.in}};
\dgfile (src/Makefilein) at (4,5) {\filenamew{src/Makefile.in}};
\dgfile (confighin) at (7,5) {\filenamew{config.h.in}};
\dgfile (configure) at (-.7,4) {\filenamew{configure}};
\uncover<all:2->{\bfile (configlog) at (5.5,2) {\filenamew{config.log}};}
\uncover<all:3->{\bfile (configstatus) at (2.5,3) {\filenamew{config.status}};}

\uncover<all:4->{\bfile (Makefile) at (1,1) {\filenamew{Makefile}};
\bfile (src/Makefile) at (4,1) {\filenamew{src/Makefile}};
\bfile (configh) at (7,1) {\filenamew{config.h}};}
\uncover<all:5->{\bfile (configcache) at (0,2) {\filenamew{config.cache}};}

\uncover<all:2>{\arr[color=red] (configure) .. controls +(10:2cm) and +(80:2cm) .. (configlog);}
\uncover<all:3->{\arr (configure) .. controls +(10:2cm) and +(80:2cm) .. (configlog);}
\uncover<all:3>{\arr[color=red] (configure) .. controls +(0:1cm) and +(90:1cm) .. (configstatus);}
\uncover<all:4->{\arr (configure) .. controls +(0:1cm) and +(90:1cm) .. (configstatus);}
\uncover<all:5>{\arr[color=red] (configure) .. controls +(-10:1.5cm) and +(80:1cm) .. (configcache);
              \arr[color=red] (configcache) .. controls +(-140:2cm) and +(-170:2cm) .. (configure);}
\uncover<all:4>{\arr[color=red] (Makefilein) .. controls +(-90:1cm) and +(180:2.8cm) .. (configstatus);
\arr[color=red] (src/Makefilein) .. controls +(-90:1cm) and +(170:3cm) .. (configstatus);
\arr[color=red] (confighin) .. controls +(-90:1cm) and +(170:3cm) .. (configstatus);

\arr[color=red] (configstatus) .. controls +(-10:3cm) and +(70:1cm) .. (Makefile);
\arr[color=red] (configstatus) .. controls +(-5:2.5cm) and +(90:2cm) .. (src/Makefile);
\arr[color=red] (configstatus) .. controls +(5:2cm) and +(90:2cm) .. (configh);
\arr[color=red] (configstatus) .. controls +(0:2cm) and +(100:.5cm) .. (configlog);
}
\uncover<all:5->{\arr (Makefilein) .. controls +(-90:1cm) and +(180:2.8cm) .. (configstatus);
\arr (src/Makefilein) .. controls +(-90:1cm) and +(170:3cm) .. (configstatus);
\arr (confighin) .. controls +(-90:1cm) and +(170:3cm) .. (configstatus);

\arr (configstatus) .. controls +(-10:3cm) and +(70:1cm) .. (Makefile);
\arr (configstatus) .. controls +(-5:2.5cm) and +(90:2cm) .. (src/Makefile);
\arr (configstatus) .. controls +(5:2cm) and +(90:2cm) .. (configh);
\arr (configstatus) .. controls +(0:2cm) and +(100:.5cm) .. (configlog);
}
\end{tikzpicture}
\footnotesize{ Взято из http://www.lrde.epita.fr/~adl/autotools.html}
\end{center}

\end{frame}



\begin{frame}
 \frametitle{Важные опции configure, требуемые GNU standard}
   Директории
    \begin{enumerate}
     \item[prefix] По дефолту \texttt{/usr/local}
      \begin{enumerate}
         \item[exec-prefix] По дефолту \texttt{prefix}
         \begin{enumerate}
            \item[bindir] По дефолту \texttt{exec-prefix/bin}
            \item[libdir] По дефолту \texttt{exec-prefix/lib}
         \end{enumerate}
         \item[includedir] \texttt{prefix/include}
         \item[datarootdir] \texttt{prefix/share}
         \begin{enumerate}
            \item[mandir]
            \item[datadir]
         \end{enumerate}
      \end{enumerate}
    \end{enumerate}

\end{frame}

\begin{frame}
 \frametitle{Еще важные опции \texttt{configure}}
 \begin{itemize}
   \item \texttt{help}
   \item \texttt{version}
   \item \texttt{with-smth}, \texttt{without-smth}
  \end{itemize}
\end{frame}


\begin{frame}
  \frametitle{Опции для make}
  \begin{itemize}
    \item \texttt{make all} (просто \texttt{make})
    \item \texttt{make install}
    \item \texttt{make uninstall}
    \item \texttt{make clean}
    \item \texttt{make distclean}
    \item \texttt{make check} Опционально
    \item \texttt{make installcheck} Опционально
    \item \texttt{make dist}
  \end{itemize}
\end{frame}

\begin{frame}
 \frametitle{Упражнение}
  \begin{itemize}
    \item Установить GNU Go в \texttt{/home/altlinux/usr/}
    \item Удалить результаты инсталляции
    \item Установить все файлы GNU Go в /home/altlinux/go
  \end{itemize}
\end{frame}

%\begin{frame}
%\frametitle{Autotools: схема работы}
%
%%This slide is copyright Alexandre Duret-Lutz, Laboratoire de Recherche et Développement de l'Epita 
%%Distributed under CC BY-SA 2.0 license
%%Minor changes Yury Adamov, EPAM 
%\only<all:1-2>{\begin{center}
%\begin{tikzpicture}
%
%\dgfile (Makefilein) at (1,5) {\filenamew{Makefile.in}};
%\dgfile (src/Makefilein) at (4,5) {\filenamew{src/Makefile.in}};
%\dgfile (confighin) at (7,5) {\filenamew{config.h.in}};
%\dgfile (configure) at (-.7,4) {\filenamew{configure}};
%\uncover<all:1>{\bfile (config.log) at (5.5,2) {\filenamew{config.log}};
%\bfile (configstatus) at (2.5,3) {\filenamew{config.status}};
%
%\bfile (Makefile) at (1,1) {\filenamew{Makefile}};
%\bfile (src/Makefile) at (4,1) {\filenamew{src/Makefile}};
%\bfile (configh) at (7,1) {\filenamew{config.h}};
%\bfile (configcache) at (0,2) {\filenamew{config.cache}};
%
%\arr (configure) .. controls +(10:2cm) and +(80:2cm) .. (configlog);
%\arr (configure) .. controls +(0:1cm) and +(90:1cm) .. (configstatus);
%\arr (configure) .. controls +(-10:1.5cm) and +(80:1cm) .. (configcache);
%\arr (configcache) .. controls +(-140:2cm) and +(-170:2cm) .. (configure);
%
%\arr (Makefilein) .. controls +(-90:1cm) and +(180:2.8cm) .. (configstatus);
%\arr (src/Makefilein) .. controls +(-90:1cm) and +(175:3.2cm) .. (configstatus);
%\arr (confighin) .. controls +(-90:1cm) and +(170:3cm) .. (configstatus);
%
%\arr (configstatus) .. controls +(-10:3cm) and +(70:1cm) .. (Makefile);
%\arr (configstatus) .. controls +(-5:2.5cm) and +(90:1cm) .. (src/Makefile);
%\arr (configstatus) .. controls +(5:2cm) and +(90:2cm) .. (configh);
%\arr (configstatus) .. controls +(0:2cm) and +(100:.5cm) .. (configlog);
%}
%\end{tikzpicture}
%\end{center}}
%\only<all:3->{\begin{center}
%\begin{tikzpicture}{-3.5cm}{3cm}{8cm}{9cm}
%\dgfile (Makefilein) at (1,5) {\filenamew{Makefile.in}};
%\dgfile (src/Makefilein) at (4,5) {\filenamew{src/Makefile.in}};
%\dgfile (confighin) at (7,5) {\filenamew{config.h.in}};
%\dgfile (configure) at (-.5,4) {\filenamew{configure}};
%\uncover<all:4->{
%\tfile (autoreconf) at (2.5,6.5) {\command{autoreconf}};
%\afile (configureac) at (-2,8) {\filenamew{configure.ac}};
%}
%\uncover<all:5->{
%\afile (Makefileam) at (1,8) {\filenamew{Makefile.am}};
%\afile (src/Makefileam) at (4,8) {\filenamew{src/Makefile.am}};
%}
%\uncover<all:4->{
%\arr (configureac) .. controls +(-90:2cm) and +(185:4cm) .. (autoreconf);
%\arr (autoreconf) .. controls +(-10:4cm) and +(100:2.7cm) .. (configure);
%\arr (autoreconf) .. controls +(5:4cm) and +(90:1cm) .. (confighin);
%}
%\uncover<all:5->{
%\arr (Makefileam) .. controls +(-90:.8cm) and +(178:3cm) .. (autoreconf);
%\arr (src/Makefileam) .. controls +(-90:.6cm) and +(170:3cm) .. (autoreconf);
%\arr (autoreconf) .. controls +(-5:4cm) and +(55:.5cm) .. (Makefilein);
%\arr (autoreconf) .. controls +(0:3cm) and +(70:.5cm) .. (src/Makefilein);
%}
%\end{tikzpicture}
%\end{center}}
%\footnotesize{ Взято из http://www.lrde.epita.fr/~adl/autotools.html}
%\end{frame}

%\begin{frame}
%\frametitle{Леденящие душу подробности \command{autoreconf}}
%%This slide is copyright Alexandre Duret-Lutz, Laboratoire de Recherche et Développement de l'Epita 
%%Distributed under CC BY-SA 2.0 license
%%Minor changes Yury Adamov, EPAM 
%\begin{tikzpicture}
%\dgfile (configure) at (-2,3) {\filenamew{configure}};
%\dgfile (confighin) at (.8,3) {\filenamew{config.h.in}};
%\dgfile (Makefilein) at (3.6,3) {\filenamew{Makefile.in}};
%\dgfile (src/Makefilein) at (6.4,3) {\filenamew{src/Makefile.in}};
%\uncover<all:1>{
%\tfile (autoreconf) at (2.5,6.5) {\command{autoreconf}};
%}
%
%\uncover<all:3->{
%\tfile (aclocal) at (-2,7) {\command{aclocal}};
%}
%\uncover<all:5->{
%\tfile (autoconf) at (-2,5) {\command{autoconf}};
%}
%\uncover<all:7->{
%\tfile (autoheader) at (1.75,5.75) {\command{autoheader}};
%}
%\uncover<all:9->{
%\tfile (automake) at (5.5,6.5) {\command{automake}};
%}
%
%
%\afile (configureac) at (-2,9) {\filenamew{configure.ac}};
%\uncover<all:4->{\dgfile (aclocalm4) at (1,8.5) {\filenamew{aclocal.m4}};}
%\afile (Makefileam) at (3.5,9) {\filenamew{Makefile.am}};
%\afile (src/Makefileam) at (6.5,9) {\filenamew{src/Makefile.am}};
%
%
%\uncover<all:1>{
%\arr (autoreconf) .. controls +(-10:4cm) and +(40:2cm) .. (configure);
%\arr (autoreconf) .. controls +(-5:4cm) and +(55:2cm) .. (confighin);
%\arr (autoreconf) .. controls +(0:4cm) and +(70:2cm) .. (Makefilein);
%\arr (autoreconf) .. controls +(5:4cm) and +(90:2cm) .. (src/Makefilein);
%
%\arr (configureac) .. controls +(-90:2cm) and +(180:4cm) .. (autoreconf);
%\arr (Makefileam) .. controls +(-120:2cm) and +(175:4cm) .. (autoreconf);
%\arr (src/Makefileam) .. controls +(-130:2cm) and +(170:4cm) .. (autoreconf);
%}
%\uncover<all:4->{
%\arr (configureac) .. controls +(-90:.5cm) and +(170:1.7cm) .. (aclocal);
%\arr (aclocal) .. controls +(10:2cm) and +(150:2cm) .. (aclocalm4);
%}
%\uncover<all:6->{
%\arr (configureac) .. controls +(-140:2cm) and +(175:2cm) .. (autoconf);
%\arr (aclocalm4) .. controls +(-120:2cm) and +(170:2cm) .. (autoconf);
%\arr (autoconf) .. controls +(-10:2cm) and +(90:1cm) .. (configure);
%}
%\uncover<all:8->{
%\arr (configureac) .. controls +(-50:2cm) and +(175:2.5cm) .. (autoheader);
%\arr (aclocalm4) .. controls +(-110:2cm) and +(170:2cm) .. (autoheader);
%\arr (autoheader) .. controls +(-10:2.5cm) and +(90:1cm) .. (confighin);
%}
%\uncover<all:10->{
%\arr (configureac) .. controls +(-40:3cm) and +(180:4cm) .. (automake);
%\arr (aclocalm4) .. controls +(-50:2cm) and +(176:2cm) .. (automake);
%\arr (Makefileam) .. controls +(-90:2cm) and +(173:2cm) .. (automake);
%\arr (src/Makefileam) .. controls +(-110:2cm) and +(170:2cm) .. (automake);
%\arr (automake) .. controls +(-10:2cm) and +(80:1cm) .. (Makefilein);
%\arr (automake) .. controls +(-5:2cm) and +(90:1cm) .. (src/Makefilein);
%}
%\end{tikzpicture}
%\end{frame}

\begin{frame}[fragile]
 \frametitle{Возможный алгоритм использования autotools}
 \begin{itemize}
  \item Запустить \texttt{autoscan}
  \item Заполнить сгенерированный шаблон \texttt{configure.scan} и сохранить как \texttt{configure.ac}
  \item Заполнить Makefile.am в каждой поддиректории
  \item Запустить \texttt{autoreconf --install --force}
  \item Починить проблемы
  \item \verb+./configure && make && make install+
 \end{itemize}
\end{frame}

\begin{frame}[fragile]
\frametitle{Пример \texttt{configure.ac}}
\begin{lstlisting}[language=sh]
AC_PREREQ(2.60)
AC_INIT([my-package],[1.0.0],[me@maintainer.org])
AC_CONFIG_AUX_DIR(autotools)

AC_CONFIG_SRCDIR(src/my_package_main.c)

AC_CONFIG_HEADER(config.h)
AC_CONFIG_FILES([Makefile src/Makefile])

AC_PROG_CC
AC_PROG_LIBTOOL

AM_INIT_AUTOMAKE([foreign -Wall -Werror])
AC_OUTPUT
\end{lstlisting}
\end{frame}

\begin{frame}[fragile]
\frametitle{Пример \texttt{Makefile.am}}
\begin{lstlisting}[language=sh]
bin_PROGRAMS = myprogram
lib_LTLIBRARIES = libmylibrary.la
myprogram_SOURCES = myprogram.c myprogram.h somefunction.c somefunction.h
myprogramm_LDADD = libmylibrary.la 
libmylibrary_la_SOURCES = mylibfun1.c mylibfun2.c libmylibrary.h
include_HEADERS = libmylibrary.h
\end{lstlisting}
\end{frame}

\begin{frame}[fragile]
\frametitle{lib LTLIBRARIES и libtool}
\begin{itemize}
 \item Разные системы -- разные расширения для динамических библиотек (
\texttt{.so, .dylib, .dll})
 \item Некоторые системы вообще не поддерживают динамическую линковку
 \item Ответ на все это -- \texttt{libtool}
 \item Кроме библиотеки с системным расширением создает \texttt{libsomething.la} с которой умеет работать на всех системах
 \item \texttt{Makefile.am} -- переменная \verb+lib_LTLIBRARY+ 
\end{itemize}
\end{frame}

\begin{frame}
 \frametitle{Упражнение}
 \begin{itemize}
  \item Создать проект типа Hello,World с отдельным файлом, который будет содержать функцию \texttt{sayhelloworld} и \texttt{sayhello(char *user)} 
  \item Создать configure.ac и Makefile.am для случая, когда hello компилируется сразу в один файл
  \item То же самое, когда компилируется в динамическую разделяемую библиотеку и конечную программу
 \end{itemize}
\end{frame}

\begin{frame}
 \frametitle{Как работают \texttt{autotools}: \texttt{m4}}
\end{frame}

\begin{frame}
 \frametitle{Альтернативы autotools}
\begin{itemize}
  \item{Основанные на make}
  \begin{itemize}
    \item CMake
    \item MakeKit
    \item qmake
  \end{itemize}
  \item{Другие}
    \begin{itemize}
      \item SCons
      \item rake, cabal, ant, etc.
    \end{itemize}
\end{itemize}
\end{frame}
