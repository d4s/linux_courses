\input{common}

\title{Введение в GNU/Linux}


%%%%%%%%%%%%%%%%%%%%%%%%%%%%%%%%%%%%%%%%%%%%%%%%%
%%%%%%%%%% Begin Document  %%%%%%%%%%%%%%%%%%%%%%
%%%%%%%%%%%%%%%%%%%%%%%%%%%%%%%%%%%%%%%%%%%%%%%%%




\begin{document}

\begin{frame}
	\frametitle{}
	\titlepage
	\vspace{-0.5cm}
	\begin{center}
	%\frontpagelogo
	\end{center}
\end{frame}


\begin{frame}
	\tableofcontents
	[hideallsubsections]
\end{frame}


\begin{frame}{Основы ОС Linux}

	\begin{block}{Вопрос}
	Почему Linux является самой популярной
	свободной операционной системой?
	\end{block}

	\pause

	\begin{block}{Ответ}
	\begin{itemize}
		\item \textcopyleft -- Copyleft
		\item ``Философия'' Unix
		\item Открытые стандарты
	\end{itemize}
	\end{block}

\end{frame}


%%%%%%%%%%%%%%%%%%%%%%%%%%%%%%%%%%%%%%%%%   
%%%%%%%%%% Content starts here %%%%%%%%%%
%%%%%%%%%%%%%%%%%%%%%%%%%%%%%%%%%%%%%%%%%

\section[Принципы]{Базовые принципы ОС Linux}

\subsection{GNU/Linux}

\mode<all>{\input{../../slides/intro/vocabulary}}

\subsection{Лицензии}

\mode<all>{\input{../../slides/intro/licenses}}

\subsection{Принципы проектирования переносимых программ}

\mode<all>{\input{../../slides/intro/unixway}}

\section{Дистрибутивы ОС Linux}

\mode<all>{\input{../../slides/intro/linux-distro}}

\section{Процесс загрузки ОС Linux}

\subsection{Этапы загрузки}

\mode<all>{\input{../../slides/intro/linux-boot}}

\subsection{Ядро Linux}

\mode<all>{\input{../../slides/intro/linux-kernel.tex}}

\subsection{Userspace}

\mode<all>{\input{../../slides/intro/initrd}}

\mode<all>{\input{../../slides/intro/init-process}}

\subsection{Практика}

\mode<all>{\input{../../slides/intro/practice01.tex}}

\end{document}
